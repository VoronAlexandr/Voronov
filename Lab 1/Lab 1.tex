\documentclass[11pt]{article}
\usepackage{amsmath,amssymb,amsthm}
\usepackage{algorithm}
\usepackage[noend]{algpseudocode} 
\usepackage[normalem]{ulem}
\usepackage{cancel}

\usepackage{hyperref} %for url
%---enable russian----

\usepackage[utf8]{inputenc}
\usepackage[russian]{babel}


% PROBABILITY SYMBOLS
\newcommand*\PROB\Pr 
\DeclareMathOperator*{\EXPECT}{\mathbb{E}}
\usepackage{fancyhdr}

% Sets, Rngs, ets 
\newcommand{\N}{{{\mathbb N}}}
\newcommand{\Z}{{{\mathbb Z}}}
\newcommand{\R}{{{\mathbb R}}}
\newcommand{\Zp}{\ints_p} % Integers modulo p
\newcommand{\Zq}{\ints_q} % Integers modulo q
\newcommand{\Zn}{\ints_N} % Integers modulo N

% Landau 
\newcommand{\bigO}{\mathcal{O}}
\newcommand*{\OLandau}{\bigO}
\newcommand*{\WLandau}{\Omega}
\newcommand*{\xOLandau}{\widetilde{\OLandau}}
\newcommand*{\xWLandau}{\widetilde{\WLandau}}
\newcommand*{\TLandau}{\Theta}
\newcommand*{\xTLandau}{\widetilde{\TLandau}}
\newcommand{\smallo}{o} %technically, an omicron
\newcommand{\softO}{\widetilde{\bigO}}  
\newcommand{\wLandau}{\omega}
\newcommand{\negl}{\mathrm{negl}} 

\newcommand{\out}[1]{\raisebox{1pt}{\sout{\phantom{#1}}}\llap{#1}}
\newcommand{\wavy}[1]{\raisebox{5pt}{\uwave{\phantom{#1}}}\llap{#1}}
% Misc
\newcommand{\eps}{\varepsilon}
\newcommand{\inprod}[1]{\left\langle #1 \right\rangle}


\newcommand{\handout}[5]{
	\noindent
	\begin{center}
		\framebox{
			\vbox{
				\hbox to 5.78in { {\bf Научно-исследовательская практика} \hfill #2 }
				\vspace{4mm}
				\hbox to 5.78in { {\Large \hfill #5  \hfill} }
				\vspace{2mm}
				\hbox to 5.78in { {\em #3 \hfill #4} }
			}
		}
	\end{center}
	\vspace*{4mm}
}

\newcommand{\lecture}[4]{\handout{#1}{#2}{#3}{Scribe: #4}{#1}}

\newtheorem{theorem}{Теорема}
\newtheorem{lemma}{Лемма}
\newtheorem{definition}{Определение}
\newtheorem{corollary}{Следствие}
\newtheorem{fact}{Факт}

% 1-inch margins
\topmargin 0pt
\advance \topmargin by -\headheight
\advance \topmargin by -\headsep
\textheight 8.9in
\oddsidemargin 0pt
\evensidemargin \oddsidemargin
\marginparwidth 0.5in
\textwidth 6.5in

\parindent 0in
\parskip 1.5ex

\setcounter{page}{54}

	
\begin{document}
	\thispagestyle{fancy}
	\rhead{CHAP. 3}
	\chead{Простые числа и их распределение}
	\lhead{\thepage}

удалены (четные 3 были удалены на предыдущем шаге). Наименьшее целое число после 3, которое ещё не было удалено, равно 5. Оно не делится ни на 2, ни на 3, иначе оно было бы вычеркнуто, следовательно, оно также является простым. Все правильные кратные 5 являются составными числами, затем мы удаляем $10, 15, 20, ...$ (некоторые из них, конечно, уже отсутствуют), сохраняя при этом само 5. Первое выживающее целое число $7$ - это простое число, поскольку оно не делится на $2$, $3$ или $5$, единственные простые числа, которые предшествуют ему. После устранения правильных кратных $7$, наибольшее простое число, меньше чем $\sqrt {100}= 10$, все остальные составные целые числа в последовательности $2, 3, 4, ..., 100$ провалились через решето. Оставшиеся целые положительные числа, а именно $2, 3, 5, 7, 11, 13, 17, 19, 23, 29, 31, 37, 41, 43,\\ 47, 53, 59, 61, 67, 71, 73 79, 83, 89, 97$, все простые числа меньше $100$.
\setlength{\parindent}{5ex}

В таблице ниже представлен результат заполненного решета. Кратные числа 2 зачёркнуты $\backslash$ ; кратные 3 зачёркнуты $\slash$ ; кратные 5 зачёркнуты — ; кратные 7 перечёркнуты $\sim$.

\begin{center}
\begin {tabular}{c c c c c c c c c c}
 &$2$&3&$\bcancel4$&5&$\bcancel{\cancel6}$&7&$\bcancel8$&$\cancel9$&$\bcancel{\out{10}}$\\
 11&$\bcancel{\cancel{12}}$&13&$\wavy{\bcancel{14}}$&$\out{\cancel{15}}$&$\bcancel{16}$&17&$\bcancel{\cancel{18}}$&19&$\bcancel{\out{20}}$\\
 $\wavy{\cancel{{21}}}$&$\bcancel{22}$&23&$\bcancel{\cancel{24}}$&$\out{25}$&$\bcancel{26}$&$\cancel{27}$&$\wavy{\bcancel{28}}$&29&$\bcancel{\out{30}}$\\
 31&$\bcancel{32}$&$\cancel{33}$&$\bcancel{34}$&$\wavy{\out{35}}$&$\bcancel{\cancel{36}}$&37&$\bcancel{38}$&$\cancel{39}$&$\bcancel{\out{40}}$\\
 41&$\wavy{\bcancel{\cancel{42}}}$&43&$\bcancel{44}$&$\cancel{\out{45}}$&$\bcancel{46}$&47&$\bcancel{\cancel{48}}$&$\wavy{49}$&$\bcancel{\out{50}}$\\
 $\cancel{51}$&$\bcancel{52}$&53&$\bcancel{\cancel{54}}$&$\out{55}$&$\wavy{\bcancel{56}}$&$\cancel{57}$&$\bcancel{58}$&59&$\bcancel{\cancel{\out{60}}}$\\
 61&$\bcancel{62}$&$\wavy{\cancel{63}}$&$\bcancel{64}$&$\out{65}$&$\bcancel{\cancel{66}}$&67&$\bcancel{68}$&$\cancel{69}$&$\wavy{\bcancel{\out{70}}}$\\
 71&$\bcancel{\cancel{72}}$&73&$\bcancel{74}$&$\cancel{\out{75}}$&$\bcancel{76}$&$\wavy{77}$&$\bcancel{\cancel{78}}$&79&$\bcancel{\out{80}}$\\
 $\cancel{81}$&$\bcancel{82}$&83&$\wavy{\bcancel{\cancel{84}}}$&$\out{85}$&$\bcancel{86}$&$\cancel{87}$&$\bcancel{88}$&89&$\bcancel{\cancel{\out{90}}}$\\
 $\wavy{91}$&$\bcancel{92}$&$\cancel{93}$&$\bcancel{94}$&$\out{95}$&$\bcancel{\cancel{96}}$&97&$\wavy{\bcancel{98}}$&$\cancel{99}$&$\bcancel{\out{100}}$\\
 
\end{tabular}
\end{center}

К этому моменту должен был возникнуть очевидный вопрос у читателя. Существует ли наибольшее простое число или простых чисел бесконечно много ? Ответ можно найти в удивительно простом доказательстве, данном Евклидом в IX книге его “Начала”. Аргумент Евклида повсеместно рассматривается как модель математической элегантности. Проще говоря, это выглядит так: при любом конечном списке простых чисел всегда можно найти простое число, не входящее в этот список ; следовательно, число простых чисел бесконечн.Фактические детали отображаются ниже.

\begin{theorem}[Евклида]
\label{th3-4}
Существует бесконечное число простых чисел.
\end{theorem}
\begin{proof}
	доказательство Евклида от противного. Пусть ${p_{1} = 2, p_{2} = 3, p _{3} = 5, p_{4} = 7,...}$ -- простые числа в порядке возрастания, и пусть что есть последнее число; назовите его $p_{n}$. Теперь рассмотрим положительное целое число \[P = p_{1}p_{2}...p_{n} + 1.\] Поскольку $P > 1$, мы можем снова применить теорему 3--2 и заключить, что P делится на некоторое простое p. Но $p_{1},p_{2},...,p_{n}$ являются единственными простыми числами, так что p должно быть равно одному из $p_{1},p_{2},...,p_{n}$. Объединяя отношение $p\mid p_{1}p_{2}...p_{n}$ с $p\mid P$, мы получим  $p\mid P - p_{1}p_{2}...p_{n}$, что эквивалентно $p\mid 1$. Единственным положительным делителем целого числа $1$ является само число $1$, и, поскольку $p > 1$, возникает противоречие. Таким образом, конечный список простых чисел не является полным, поэтому число простых чисел бесконечно.
	\end{proof}

\thispagestyle{fancy}
\rhead{\thepage}
\chead{Решето Эратосфена}
\lhead{Sec.3-2}	

Интересно отметить, что при формировании целых чисел \[P_{k} = p_{1}p_{2}...p_{n} + 1,\] первые пять, а именно : \[P_{1} = 2 + 1 = 3,\] 
\[P_{2} = 2\cdot3 + 1 = 7,\] \[P_{3} = 2\cdot3\cdot5 + 1 = 31,\] \[P_{4} = 2\cdot3\cdot5\cdot7 + 1 = 211,\]\[P_{5} = 2\cdot3\cdot5\cdot7\cdot11 + 1 = 2311,\] простые числа. Тем не мение, \[P_{6} = 59\cdot509 P_{7} = 19\cdot97\cdot277, P_{8} = 347\cdot27953\] не простые числа. Вопрос, на который ответ неизвестен, состоит в том, существует ли бесконечно много $k$, для которого $P_{k}$ является простым.Так же неизвестно, бесконечно ли много составных $P_{k}$

Теорема Евклида слишком важна для нас, чтобы довольствоваться одним доказательством. Вот вариант рассуждения: сформировать бесконечную последовательность натуральных чисел:	\footnote{Многострочные уравнения корректно оформлять через окружения align или equation \url{https://www.overleaf.com/learn/latex/aligning\_equations\_with\_amsmath}}

\begin{gather*} 
n_{1} = 2\\
n_{2} = n_{1} + 1,\\
n_{3} = n_{1}n_{2} + 1,\\
n_{4} = n_{1}n_{2}n_{3} + 1,\\
\vdots\\
n_{k} = n_{1}n_{2}...n_{k-1} + 1,\\
\end{gather*}

Поскольку каждое $n_{k}> 1$, каждое из этих целых чисел делится на простое число. Но никакие два не могут иметь один и тот же простой делитель. Чтобы увидеть это, пусть $d = \text{НОД} (n_{i},n_{k})$, где $i<k$. Тогда d делитель $n_{i}$, следовательно, он делитель $n_{1}n_{2}...n_{k-1}$. Т.к. $d\mid n_{k}$, Теорема 2-2(7) говорит нам, что $d\mid n_{k}-n_{1}n_{2}...n_{k-1}$ или $d \mid 1$. Подразумевается, что $d = 1$, и поэтому целые числа $n_{k}(k = 1,2,...)$ попарно взаимно просты. Смысл, который мы хотим подчеркнуть, состоит в том, что существует столько же простых чисел, сколько целых чисел $n_{k}$, а именно бесконечно много.

Пусть $p_{n}$ обозначает n простых чисел в их естественном порядке». Доказательство Евклида показывает, что оценка скорости роста $p_{n}$ принадлежит промежутку \[p_{n+1} \leq p_{1}p_{2}...p_{n}+1 < p_{n}^{n}+1 .\] Например, когда n = 3, неравенство утверждает, что \[7 = p_{4} < p_{3}^{3} + 1 = 5^{3} + 1 = 126\] Видно, что эта оценка экстравагантна. Более чёткое ограничение на размер $p_{n}$ приведено в
\thispagestyle{fancy}
\rhead{CHAP. 3}
\chead{Простые числа и их распределение}
\lhead{\thepage}
\begin{theorem}
	\label{th3-5}
	Если $p_{n}$ - это n-ое простое число, то $p \leq 2^{2n-1}$.
\end{theorem}
\begin{proof}
Продолжим индукцию по $n$ , причём утверждённое неравенство явно верно при $n = 1$. В качестве гипотезы индукции мы предполагаем, что $n > 1$ и что результат справедлив для всех целых чисел вплоть до $n$. Тогда \[p_{n+1} \leq p_{1}p_{2}...p_{n}+1 \leq 2 \cdot 2^{2} ...2^{2n-1} + 1 = 2^{1+2+2^{2}+...+2^{n-1}}+1 \]
Вспоминая тождество $1 + 2 + 2^{2}+ … + 2^{n-1} = 2^{n}  - 1$ , получим \[p_{n+1} \leq 2^{2^{n}-1}+1\]
Но $1 \leq 2^{2^{n}-1}$ для любого $n$; откуда \[p_{n+1} \leq 2^{2^n-1} + 2^{2^n-1} = 2 \cdot 2^{2^{n}-1} = 2^{2^{n}},\] Где завершается шаг индукции и аргумента.
\end{proof}
	У теоремы~\ref{th3-5} есть интересное следствие. 
	\begin{corollary}
		Для любого $n \geq 1$ существуют по крайней мере n+1 простых чисел, меньших, чем $2^{2^{n}}$
	\end{corollary}
\begin{proof}
	Из теоремы следует, что $p_1,p_2,...,p_{n+1}$ меньше $2^{2^{n}}$
\end{proof}
\section*{\center {Задания 3.2}}
\begin{enumerate}
	\item Определите, является ли целое число 701 простым, проверяя все простые числа $p \leq \sqrt{701}$ как возможные делители. Сделайте то же самое для целого числа $1009$.
	\item Используя Решето Эратосфена, получите все простые числа от $100$ до $200$.
	\item Предположим, что $p$ не делится на $n$ для всех простых чисел $p \leq \sqrt[3]{n}$, покажите, что $n$ является либо простым числом, либо произведением двух простых чисел. [Подсказка : Предположим противное: $n$ содержит хотя бы три простых разложения.]
	\item Установите следующие факты:
	\begin{enumerate}
		\item $\sqrt{p}$ иррационально для любого простого числа $p$
			\item Если $a > 0$ и $\sqrt[n]{a}$ рационально, то $\sqrt[n]{a}$ должно быть целым числом
				\item При $n \geq 2$ $\sqrt[n]{n}$ иррационально. [Подсказка : используйте тот факт, что $2^{n}>n$]
	\end{enumerate}
\item Покажите, что любое составное трёхзначное число должно иметь простой коэффициент, меньший или равный $31$.
\item Заполните все недостающие детали в этом наброске доказательства бесконечности простых чисел : предположим, что количество простых чисел конечно, скажем, $p_{1},p_{2},...,p_{n}$.Пусть A - произведение любого $r$ этих простых чисел и положим , что $B =p_{1}p_{2}...p_{n}/A$. Тогда каждое $p_{k}$ является делителем либо А, либо В, но не обоих. Поскольку $A+B > 1, A+B$ имеет простой делитель, отличный от любого из $p_{k}$, а это противоречие
\item Измените доказательство Евклида, что существует бесконечно много простых чисел, предполагая существование наибольшего простого числа p и используя целое число ${N = p!+ 1}$
\item Дайте ещё одно доказательство бесконечности простых чисел, предполагая использование только конечного числа простых чисел, скажем ${p_{1},p_{2},...,p_{n},}$ и с использованием целого числа \[N = p_2p_3...p_n + p_1p_3...p_n +...+ p_1p_2...p_{n-1}\]чтобы прийти к противоречию.
\item Докажите, что если $n > 2$, то существует простое число $p$, удовлетворяющее условию $n < p < n!$. [Подсказка: если $n! - 1$ не простое число, то оно имеет простой делитель $p$ ; $p \leq n$ подразумевает, что $p | n!$ приводя к противоречию.]
\item Если $p_n$ обозначает n -ое простое число , покажите, что ни одно из целых чисел ${P_n = p_{1}p_{2}...p_{n}+1}$ не является квадратом. [Подсказка: Каждое $P_n$ имеет форму $4k + 3$.]
\end{enumerate}
\thispagestyle{fancy}
\rhead{\thepage}
\chead{Решето Эратосфена}
\lhead{Sec.3-2}
\end{document}
